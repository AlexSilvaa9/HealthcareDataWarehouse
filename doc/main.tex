\documentclass{article}
\usepackage{fullpage}

%load needed packages
\usepackage{graphicx}
\usepackage{array}
\usepackage{booktabs}
\usepackage[utf8]{inputenc}
\usepackage[T1]{fontenc}
\usepackage{url}
\usepackage[spanish]{babel} % Paquete para el idioma español
\usepackage{float}  % Necesario para [H]
\usepackage{listings}
\usepackage{xcolor}

\definecolor{codegreen}{HTML}{5AB2FF}
\definecolor{morado}{HTML}{AD88C6}
\definecolor{BG}{HTML}{EEEEEE}
\definecolor{azul}{HTML}{4D869C}
\definecolor{sqlblue}{HTML}{FF8C00} % Color para las palabras clave SQL

% Estilo para DDL
\lstdefinestyle{ddlstyle}{
	language=SQL,
	backgroundcolor=\color{BG},
	commentstyle=\color{codegreen},
	basicstyle=\ttfamily\small,
	keywordstyle=\color{azul},
	stringstyle=\color{morado},
	showstringspaces=false,
	breaklines=true,
	frame=shadowbox,
	numbers=left,
	numberstyle=\tiny\color{gray},
	captionpos=b,
}

% Estilo para SQL
\lstdefinestyle{sqlstyle}{
	language=SQL,
	backgroundcolor=\color{BG},
	commentstyle=\color{codegreen},
	basicstyle=\ttfamily\small,
	keywordstyle=\color{sqlblue}, % Color diferente para palabras clave SQL
	stringstyle=\color{morado},
	showstringspaces=false,
	breaklines=true,
	frame=shadowbox,
	numbers=left,
	numberstyle=\tiny\color{gray},
	captionpos=b,
}

\begin{document}



% Portada
\begin{titlepage}
	\centering
	\vspace*{3cm}
	
	% Título destacado
	{\Huge \textbf{Diseño y Explotación de un almacén de 
			UCI Sanitaria}\\[0.5cm]}
	
	% Espacio y logotipo (si lo tienes, por ejemplo el logo de tu universidad)
	\vspace{2cm}
	\includegraphics[width=0.3\textwidth]{images/uma_logo.jpg}\\[1cm]
	
	% Nombre del autor
	{\LARGE \textbf{Alejandro Silva Rodríguez}\\[0.5cm]}
	{\LARGE \textbf{Marta Cuevas Rodríguez}\\[0.5cm]}
	{\large \textit{Almacenes De Datos}\\
		Universidad de Málaga\\
		}
	
	\vfill
	
	% Fecha en la parte inferior de la página
	{\large Septiembre 2024}
\end{titlepage}

% indice
\tableofcontents

\newpage
\section{Introducción}
\label{sec:introduccion}
Este proyecto tiene como objetivo desarrollar un almacén de datos enfocado en analizar el gasto en medicamentos para pacientes ingresados en la Unidad de Cuidados Intensivos (UCI) en hospitales de EE.UU. Utilizando una base de datos proporcionada por el Instituto de Tecnología de Massachusetts (MIT) \cite{eicu_crd}, que contiene información sobre síntomas, diagnósticos, y tratamientos, se modelará el almacén específicamente en torno a los costos de los fármacos administrados a los pacientes críticos.

La selección de este hecho permite un análisis detallado de los patrones de consumo y gasto en medicamentos, lo cual puede aportar información valiosa para la gestión de recursos en la UCI, mejorar la eficiencia en los tratamientos y optimizar los presupuestos hospitalarios.


\section{Diseño Conceptual}
\label{sec:diseno_conceptual}
\subsection{Descripción del Modelo Conceptual}

El modelo conceptual de nuestro almacén de datos tiene como hecho principal el \textbf{Gasto en medicamentos} en la UCI del hospital, y como medida la \textbf{cantidad} de gasto.
\\ 

Este modelo está compuesto por cinco dimensiones, descritas a continuación:

\begin{enumerate}
	\item \textbf{Paciente}: La dimensión \textit{Paciente} presenta una jerarquía múltiple con dos subjerarquías principales: \textit{Alergia} y \textit{Tratamiento}. 
	\item \textbf{Tiempo}: La dimensión \textit{Tiempo} se estructura con \textit{Año} como su único atributo.
	\item \textbf{Hospital}: Esta dimensión incluye los atributos \textit{Región} y \textit{Hospital}, permitiendo agrupar los datos por área geográfica y hospital específico.
	\item \textbf{Ingreso}: La dimensión \textit{Ingreso} se compone del atributo \textit{Ingreso}, que identifica cada episodio de admisión a la UCI.
	\item \textbf{Medicamento}: En esta dimensión, el atributo \textit{Medicamento} permite categorizar los datos en función del fármaco administrado.
\end{enumerate}



\section{Diseño Lógico}
\label{sec:diseno_logico}
\subsection{Transformación a Modelo en Copo de Nieve / Constelación}
Describe cómo el diseño conceptual se convierte en un modelo lógico de tipo copo de nieve o constelación. Incluye un diagrama del diseño lógico, destacando las relaciones entre las tablas y el esquema de la dimensión tiempo denormalizada.

\subsection{Definición de Claves}
Define las claves primarias y las claves externas de cada tabla en el esquema lógico, explicando cómo cada clave se selecciona y se asocia a las entidades correspondientes.

\section{Creación de Tablas en SQL Server}
\label{sec:creacion_tablas}
\subsection{Generación del Código DDL}
Presenta el código DDL para la creación de las tablas del almacén en SQL Server. Explica las instrucciones y pasos necesarios para ejecutar este código correctamente, incluyendo cualquier configuración especial o instrucción adicional necesaria.

\subsection{Instrucciones para Crear el Almacén}
Proporciona una guía para desplegar el almacén de datos en SQL Server usando el código DDL, detallando los pasos a seguir y cualquier configuración de permisos o ajustes adicionales.

\section{Dificultades Encontradas}
\label{sec:dificultades_encontradas}
Enumera y describe las dificultades enfrentadas durante el desarrollo del proyecto. Explica cómo se resolvieron estos problemas o, en caso de no haber encontrado complicaciones, indícalo claramente.

\section{Conclusión}
\label{sec:conclusion}
Resumen de los principales resultados del proyecto, con un breve comentario sobre la utilidad del almacén de datos desarrollado y cómo este se alinea con los objetivos de la asignatura.


\newpage
\section{Acceso al Repositorio}

Toda la información adicional, incluyendo el código fuente y la documentación completa de este proyecto, está disponible en el repositorio de GitHub \cite{silva2024github}.

% Incluir la bibliografía
\bibliographystyle{plain}  % Estilo de la bibliografía (por ejemplo, plain, alpha, ieee, etc.)
\bibliography{bibli}  % Nombre del archivo .bib sin la extensión

\end{document}
